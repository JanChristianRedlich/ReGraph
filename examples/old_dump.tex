
\begin{center}
	\hspace{-20pt}
	\begin{tikzcd}
		L \arrow[d, tail] & P \arrow[l] \arrow[r] \arrow[d, tail] & R \arrow[d, tail] \\
		G & G^- \arrow[l] \arrow[ddr]\arrow[r] & G^+ \arrow[ddr] & \\
		% & & P' \arrow[r] \arrow[tail, d] & R' \arrow[tail, d] \\
		& \\
		& & G' \arrow[r] & G'^+
	\end{tikzcd}
	?
\end{center}


\begin{center}
	\begin{tikzcd}
		G' \arrow[ddr] & G'^- \arrow[l] \arrow[ddr] \\
		& L \arrow[d, tail] & P \arrow[l] \arrow[r] \arrow[d, tail] & R \arrow[d, tail] \\
		& G & G^- \arrow[l] \arrow[r] \arrow[ddr] & G^+ \arrow[ddr] \\
		& \\
		& & & G' \arrow[r] & G'^+
	\end{tikzcd}
\end{center}

If I define $T \leftarrow R_T \rightarrow R$, I define some constraints on propagation to $G' \in suc(G)$.

\begin{tikzcd}
	T \arrow[dr, bend right] \arrow[ddr, bend right] & R_T \arrow[l] \arrow [r] & R \arrow[dl, bend left] \arrow[ddl, bend left] \\
	& T' \arrow[d, dotted]\\
	& T'_R \\
\end{tikzcd}
\begin{tikzcd}
	& H' \arrow[d, dotted] \arrow[ddl, bend right] \arrow[ddr, bend left] \\
	& H_P \arrow[dl, bend right] \arrow [dr, bend left] \\
	H \arrow[r] & G & P \arrow[l]
\end{tikzcd}



\begin{enumerate}
    

\item Consider some graph $G$ in a graph hierarchy and a set of its immediate successors (graphs typing $G$) as in the figure below:

\begin{center}
    \begin{tikzcd}    
            &     & G \arrow[dll, start anchor={[xshift=-1.5ex]}] \arrow[dl] \arrow[d, phantom, "\ldots"] \arrow[dr] \arrow[drr, start anchor={[xshift=1.5ex]}] \\
        T_1 & T_2 & \ldots & T_{m-1} & T_m \\
    \end{tikzcd}
\end{center}

We want to perform a type-preserving rewriting $G \rightsquigarrow G'$ with a rule $r: L \leftarrow P \rightarrow R$ in this context, namely we want $G'$ to be typed by $T_1, T_2, \ldots, T_n$ after rewriting, and for this purpose typing of $R$ by every such successor should be provided as discussed in the previous section. 

\begin{remark}
    In general, we allow hierarchies to have partial typing homomorphisms (more details can be found in the appendix \ref{app:AppendixB}), in which case it is not required to type the right-hand side of the rule $R$. If the rewriting rule creates new nodes which are not typed by some $T_k$ and if typing $G \rightarrow T_k$ was total, as the result of  rewriting this typing becomes partial.
\end{remark}

\item Consider a set of graphs typed by $G$ (both immediate predecessors and graphs typed transitively via composition of homomorphisms along the path to $G$) $H_1, H_2, \ldots, H_n$ as on the figure below:

\begin{center}
    \begin{tikzcd}
        H_1 \arrow[drr, end anchor={[xshift=-1.5ex]}] & H_2 \arrow[dr] & \ldots \arrow[d, phantom, "\ldots"] & H_{n-1} \arrow[dl] & H_n \arrow[dll, end anchor={[xshift=1.5ex]}] \\    
            &     & G \\
    \end{tikzcd}
\end{center}

After rewriting $G \rightsquigarrow G'$ we would like to \textit{propagate} the changes in $G$ to all the graphs $H_1, H_2, \ldots, H_n$ typed by it. Specifically, to preserve the coherence of the hierarchy we want to remove and clone all the nodes (and edges) that map to the nodes in $G$ which are respectively removed or cloned. Adds and merges produced by the application of the rule do not affect the graphs typed by $G$ in the hierarchy.

For every graph $H_k$ typed by $G$ (immediately or transitively) and the span $G \leftarrow G^- \rightarrow G'$ produced as the result of sesqui-pushout rewriting procedure, propagation of deletions and clones from $G$ to $H_k$ can be performed with the pullback from $H_k \rightarrow G \leftarrow G^-$ as illustrated on the diagram below (here and further in this section we denote with green arrows the newly constructed homomorphisms to distinguish them from the homomorphisms initially given in the hierarchy and denoted with black arrows):

% Recall that the graph $G^{-}$ corresponds to the result of deletion of nodes/edges and cloning of nodes, and the graph $G'$ corresponds to the one of addition of nodes/edges and merging of nodes specified by a rule. 
\begin{center}
    \begin{tikzcd}
        H_k \arrow[d] & H_k^{-} \arrow[l, green!80!black] \arrow[d, green!80!black] \\
        G & G^- \arrow[l, green!80!black] \arrow[r, green!80!black] &  G' \\
    \end{tikzcd}
\end{center}

The graph $H_k^-$ corresponds to the result of deletions and cloning of all the nodes in $H_k$ which are typed by the nodes in $G$ that were either deleted or cloned in the course of rewriting.

% Propagation of adds and merges is not possible in general case. To perform it the final pullback complement should be found from $H^-_k \rightarrow G^- \rightarrow G'$, and it exists only if $G^- \rightarrow G'$ is a monomorphism ($G^- \rightarrowtail G$), which is not always the case by construction of $G \leftarrow G^{-} \rightarrow G'$. 


\item For all $H_l$ that type $H_k$, which has been affected by the previously specified propagation, the typing is updated with the composition $(H_k \rightarrow H_l) \circ (H_k^- \rightarrow H_k)$ as shows the diagram below:

\begin{center}
    \begin{tikzcd}
         & H_k \arrow[dl]  \arrow[d] & H_k^{-} \arrow[l, green!80!black] \arrow[d, green!80!black] \arrow[dll, red, bend right=50] \\
        H_l & G & G^- \arrow[l, green!80!black] \arrow[r, green!80!black] &  G' \\
    \end{tikzcd}
\end{center}

Consider a graph $H_k$ typed by $G$ via some homomorphism $f_1$ and a graph $H_l$ typed by $H_k$ via $f_2$ (illustrated on the diagram below). $H_l$ is typed by $G$ not directly, but transitively via the homomorphism constructed from the composition $f_1 \circ f_2$. In this case propagation of changes in $G$ is performed similarly by finding the pullback from $H_l \rightarrow G \leftarrow G^-$. However, to preserve the initial graph structure of the hierarchy, a homomorphism $f'_2: H_l^- \rightarrow H_k^-$ should be found. Due to the fact that $f_1 \circ f_2 \circ h_2 = g \circ u$ and the universal property of the pullback $H_k \leftarrow H_k^- \rightarrow G^-$ from $H_k \rightarrow G \leftarrow G^-$, there exists a unique homomorphism $f'_2$ as on the diagram below:

\vspace{-20pt}
\begin{center}
    \begin{tikzcd}
        H_l \arrow[d, "f_2"] \arrow[dd, bend right, green!80!black] & H_l^- \arrow[l, "h_2", green!80!black] \arrow[dd, bend left, "u", green!80!black] \arrow[d, dashed, "f_2'"', red] \\
        H_k \arrow[d, "f_1"] & H_k^{-} \arrow[l, "h_1", green!80!black] \arrow[d, "f_1'"', green!80!black] \\
        G & G^- \arrow[l, "g", green!80!black]  \\
    \end{tikzcd}
\end{center}
\vspace{-20pt}

A special case arises when we propagate the changes to a graph whose predecessor has already been updated. For example, consider a hierarchy with the following structure: $H_k \rightarrow G$, $H_l \rightarrow G$ and $H_k \rightarrow H_l$ (denoted with black arrows on the diagram below). The changes to $H_k$ have already been propagated (and therefore $H_k^-$ has been constructed). Now we would like to propagate the changes to $H_l$, and to find an updated edge $H_k^- \rightarrow H_l^-$. $H_l^-$ is constructed the same way as previously by finding the pullback $H_l \leftarrow H_l^- \rightarrow G^-$ from $H_l \rightarrow G \leftarrow G^-$. From the fact that $g \circ f_1'= f_2 \circ f_3 \circ h_1$ and the universal property of the pullback follows that there exists a unique homomorphism $f_3': H_k^- \rightarrow H_l^-$ (denoted with red dashed arrow on the diagram).

\vspace{-10pt}
\begin{center}
    \begin{tikzcd}
        & H_k^- \arrow[dl, green!80!black, "h_1"'] \arrow[ddr, green!80!black, "f_1'", near start] \arrow[rr, dashed, red, "f_3'"] & & H_l^- \arrow[dl, green!80!black, "h_2"'] \arrow[ddl, green!80!black, "f_2'"] \\
        H_k \arrow[dr, "f_1"'] \arrow[rr, "f_3"] &   & H_l \arrow[dl, "f_2"']\\
            & G & G^- \arrow[l, green!80!black, "g"] \\
    \end{tikzcd}
\end{center}
\vspace{-20pt}

\end{enumerate}

The high-level pseudocode of propagation procedure is presented in the appendix \ref{app:AppendixC}. Starting from the initial rewritten graph $G$ this procedure traverses a hierarchy in the breadth-first manner (in the reverse direction of the edges). Traversal propagates the changes to  all the graphs transitively typed by $G$ and updates all the edges affected by the propagation.



% \subsubsection*{$\mu$-formulae}

% \subsubsection{Relations}

% Relations between two graphs $G_1=(V_1, E_1,)$ and $G_2 = (V_2, E_2, )$ is defined as a set of pairs $R \subseteq V_1 \times V_2$

% Can also be represented as a span $G_1 \leftarrow G_{12} \rightarrow G_2$



\subsection{Propagation up}



\subsection{Propagation down}

Propagation up:

typing of the right-hand side

\begin{tikzcd}
	L \arrow[d, tail] & P \arrow[l] \arrow[r] \arrow[d, tail] & R \arrow[d, tail] \arrow[dd,  bend left=50] \\
	G \arrow[d] & G^- \arrow[l] \arrow[r] & G^+ \arrow[d, dashed, red]\\
	H \arrow[rr] &     & H' \\ 
\end{tikzcd}

\begin{tikzcd}
	& L \arrow[d, tail] & P \arrow[l] \arrow[r] \arrow[d, tail] & R \arrow[d, tail] \arrow[dd,  bend left=50] \\
	G_2 \arrow[dr] \arrow[drrr, red, bend right=60] & G_1 \arrow[d] & G_1^- \arrow[l] \arrow[r] & G_1^+ \arrow[d, dashed, red]\\
	& H \arrow[rr] &     & H' \\ 
\end{tikzcd}


\begin{tikzcd}
	L \arrow[d] & P \arrow[l] \arrow[r] & R \arrow[d] \arrow[dd, bend left=50]\\ 
	H \arrow[rr] \arrow[d] && H' \arrow[d, red, dashed] \\
	I \arrow[rr] && I' \\
\end{tikzcd}


\begin{tikzcd}
	L \arrow[d] & P \arrow[l] \arrow[r] & R \arrow[d] \arrow[ddl, bend left=70]\\ 
	H \arrow[rr] \arrow[dr] && H' \arrow[dl, red, dashed] \\
	& I & \\
\end{tikzcd}

%\subsection{Relations update}


%\begin{tikzcd}
%	L \arrow[d] & P \arrow[l] \arrow[r] & R \arrow[d] \arrow[ddl, bend left=70]\\ 
%	H \arrow[rr] \arrow[dr] && H' \arrow[dl, red, dashed] \\
%	& I & \\
%\end{tikzcd}



%\section{Conclusions}
